\chapter{UNITE}
\label{chap:unite}

This chapter discusses the main tool called \textit{Understanding 
Non-functional Intentions via Testing and Experimentation (UNITE)} in the CUTS 
analysis tool set. 

\section{Overview}
\label{sec:unite-overview}

UNITE is a method and a tool for analyzing system execution 
traces and validating QoS properties. Its primary purpose
is to support validation for distributed software systems,
but it can be used to analyze QoS properties of any software
system which generates a system execution trace. 
UNITE analyzes distributed system QoS properties such as 
end-to-end response time, service time and throughput using system 
execution traces. System execution trace is a collection of log messages.
These log messages are related to an execution of the system during a user 
decided time period. 

The system execution traces are collected 
using CUTS logging facilities as described in Chapter~\ref{chap:logging}.
CUTS uses SQLite (\url{http://www.sqlite.org}) 
to store the system execution trace. SQLite is a very lightweight 
database management system, which have a very flat file system. For example 
the generated system execution trace is contained in a single file and when 
the distributed system testers want to analyze the QoS properties they 
just need to show the location of this file to UNITE with a configuration 
file. 

This chapter first covers main concepts in UNITE in Section ~\ref{sec:unite-concepts}. 
Then it describes main configuration files of UNITE in  Section ~\ref{sec:unite-config}. 
Finally it shows how to use UNITE as a command line tool to analyze 
a given system execution trace in Section ~\ref{sec:unite-invoke}.

\subsection{Concepts in UNITE}
\label{sec:unite-concepts}

Table ~\ref{table:execution-trace} shows a sample system execution trace 
as stored in a database for offline QoS analysis. The most important 
data for UNITE is stored in the \textit{Message} column of the table. 
All the concepts describes below uses the log messages in the \textit{Message} 
column of the following table.

\begin{table}[h, captionpos=b]
  \caption{An example system execution trace }
  \label{table:execution-trace}
  \begin{tabular}{lcccl}
  \hline
  \textbf{ID} & \textbf{Time of Day} & \textbf{Hostname} & \textbf{Severity} & \textbf{Message} \\
  \hline
  10 & 2011-02-25 05:15:55 & sender.cs.iupui.edu & INFO & Config: sent event 5 at 120394455 \\
  11 & 2011-02-25 05:15:55 & sender.cs.iupui.edu & INFO & Planner: sent event 6 at 120394465 \\
  12 & 2011-02-25 05:15:55 & receiver.cs.iupui.edu & INFO & Planner: received event 5 at 120394476 \\
  13 & 2011-02-25 05:15:55 & sender.cs.iupui.edu & INFO & Config: sent event 7 at 120394480 \\
  14 & 2011-02-25 05:15:55 & receiver.cs.iupui.edu & INFO & Effector: received event 6 at 120394488 \\
  15 & 2011-02-25 05:15:55 & receiver.cs.iupui.edu & INFO & Planner: received event 7 at 120394502 \\
  \end{tabular}
\end{table}


Before start using UNITE following concepts need to be understood.

\begin{itemize}
  \item \textbf{Log Formats} - Each log message in the system execution 
  trace is constructed from a well defined format- called a \textit{log format}.
  \textit{Log Formats} are high-level constructs that capture both constant 
  and variable portions of individual, but similar log messages in a system 
  execution trace. For example Listing~\ref{listing:log-format}  shows 
  two different log formats which capture the log messages in 
  the above system execution trace.
  
  \begin{lstlisting}[label=listing:log-format, caption=Log formats for QoS analysis.,
   captionpos=b]
   LF1: {STRING cmp_id}:sent event {INT event_id} at {INT sent}
   LF2: {STRING cmp_id}:received event {INT event_id} at {INT recv}
  \end{lstlisting}
  
  The variable portions of the log format is specified inside place holders ({}).
  The information captured in the variable portions of a log format represent 
  metrics that  are very useful in QoS analysis. In the above example the 
  distribution of the values for the variable \textit{event\_id}, with the time 
  variables \textit{sent, recv } can be used to determine the latencies of 
  different events. The variable portion of a log format consists of a variable 
  type and a variable name. Table~\ref{table:data-types} shows the data types 
  supported by UNITE.
  
 \begin{table}[h, captionpos=b]
  \centering
  \caption{UNITE data types}
  \label{table:data-types}
  \begin{tabular}{lccl}
  \hline
  \textbf{Type} & \textbf{Description} \\
  \hline
  INT & Integer data type \\
  LONG & Long data type \\
  STRING & String data type (There cannot be spaces) \\
  FLOAT & Floating-point data type \\
  \hline
  \end{tabular}
\end{table}
  
  \item \textbf{Log Format Relations} - The other main concept in UNITE 
  is Log Format Relations. Log Format Relations define a cause-effect 
  relationship among two log formats. For example in the above example
  \texttt{LF1} represents a sent event in the distributed system. And \texttt{LF2} 
  represents a receive event in the system. The sent event always 
  happens before the receive event. There for the log message corresponds 
  to a particular sent event occurs before the corresponding receive event.
  We say \texttt{LF1} is the cause log format and \texttt{LF2} is the effect log format. In 
  order to relate the two log messages distributed system testators can use 
  the \textit{event\_id} variable as follows.
  
  \texttt{LF1.event\_id = LF2.event\_id}
  
  \item \textbf{QoS analysis expressions} - After specifying Log formats and 
  Log format relations distributed system testers have to specify a QoS analysis 
  equation to get the results. For example in the above example in order to 
  get the average response time following equation can be used.
  
  \texttt{AVG(LF2.recv - LF1.sent)}
  
\end{itemize}

The next section ~\ref{sec:unite-config} shows how to specify above 
mentioned details using the configuration files in UNITE.

\section{Configuring UNITE}
\label{sec:unite-config}

UNITE uses two xml configuration files called, \textit{datagraph file} and 
\textit{unite file} for the configurations. The \textit{datagraph file} is used to 
specify the details of log formats and relations. The \textit{unite file} 
contains a reference to the \textit{datagraph file} and is used for the 
QoS analysis.

Listing~\ref{listing:example.datagraph} shows a sample \textit{datagraph file}. 

\lstinputlisting[label=listing:example.datagraph,
  caption=An example configuration file for UNITE datagraph.,
  captionpos=b,numbers=left]{analysis/example.datagraph}

\subsection{Datagraph Schema Definition}
\label{sec:datagraph}

This section discusses the details of the UNITE datagraph configuration 
based on its underlying XML Schema Definition.

\xmltag{<cuts:datagraph>}
This is the main tag for the datagraph configuration document. Its 
XML namespace must have the \url{http://cuts.cs.iupui.edu} definition. If it does
not have this definition, then UNITE will not execute further.

\xmltag{<name>}

This tag contains the name of the datagraph. This tag does not have 
any child tags.

\xmltag{<adapter>}

This tag contains the location of an external adapter module required for 
for UNITE to correctly analyze the QoS properties. This tag is not 
mandatory. It only needs when the system execution trace does not 
contain required properties for the QoS analysis. See section () SETAF 
for more details.

\xmltag{<logformats>}

This tag is a container for set of log formats that will be used in QoS 
analysis process. It may contain any number of logformat elements. 
Log formats can be specified in any order .In general, cause log format 
is specified first. Then its effect log format is specified. 
The following is a list of child tags: \texttt{<logformat>}.

\xmltag{<logformat>}

This tag contains all the details about a particular log format. It has the following 
attributes.
\begin{table}[h]
  \begin{tabular}{lcccl}
  \hline
  \textbf{Name} & \textbf{Type} & \textbf{Default Value} & \textbf{Required} & \textbf{Description} \\
  \hline
  id & String  & & Yes & Name of the log format \\
  \end{tabular}
\end{table}

\noindent The following is a list of child tags:
\texttt{<value>}, \texttt{<relations>}.

\xmltag{<value>}
This tag contains the actual value of the log format. Value of a log 
format represent set of log messages in the system execution trace 
which have constant and variable parts as described in section (ref) 
This tag does not have any child tags.

\xmltag{<relations>}

This tag contains set of relations, in which the parent log format is 
involved in as the cause log format.
The following is a list of child tags: \texttt{<relation>}.

\xmltag{<relation>}

This tag contains all the details about a particular relation. Such 
as the effect log format and cause-effect variables. There can be 
more than one cause-effect variable pairs. It has the following 
attributes.
\begin{table}[h]
  \begin{tabular}{lcccl}
  \hline
  \textbf{Name} & \textbf{Type} & \textbf{Default Value} & \textbf{Required} & \textbf{Description} \\
  \hline
  effectref & String  & & Yes & The id of the effect log format in this relation \\
  \end{tabular}
\end{table}

\noindent The following is a list of child tags:
\texttt{<causality>}.

\xmltag{<causality>}

This tag represent the relationship between a variable in the 
cause log format and a variable in the effect log format. It has 
the following attributes.
\begin{table}[h]
  \begin{tabular}{lcccl}
  \hline
  \textbf{Name} & \textbf{Type} & \textbf{Default Value} & \textbf{Required} & \textbf{Description} \\
  \hline
  cause & String  & & Yes & The name of the cause variable \\
  effect  & String  & & Yes &  The name of the effect variable \\   
  \end{tabular}
\end{table}

Therefore the elements mentioned above represent the datagraph 
for a partiucar system execution trace. Depending on the QoS 
properties being analyzed a particular system execution trace 
may have more than one datagraphs. These different datagraphs 
may have different log formats and cause-effect relations.

In general the location of the \textit{datagraph file} is specified in the 
\textit{unite file}. Listing~\ref{listing:example.unite} shows a sample 
\textit{unite file}. 

\lstinputlisting[label=listing:example.unite,
  caption=An example configuration file for UNITE.,
  captionpos=b,numbers=left]{analysis/example.unite}

\subsection{Unite Schema Definition}
\label{sec:unite-scema}

This section discusses the details of the UNITE configuration 
based on its underlying XML Schema Definition.

\xmltag{<cuts:test>}
This is the main tag for the unite configuration document. Its 
XML namespace must have the \url{http://cuts.cs.iupui.edu} definition. If it does
not have this definition, then UNITE will not execute further.

\xmltag{<name>}

This tag contains the name of the unite file. This tag does not have 
any child tags.

\xmltag{<datagraph>}

This tag specify the location of the associated datagraph for the 
QoS analysis. This tag has the following attributes.

\begin{table}[h]
  \begin{tabular}{lcccl}
  \hline
  \textbf{Name} & \textbf{Type} & \textbf{Default Value} & \textbf{Required} & \textbf{Description} \\
  \hline
  location & String  & & Yes & The location of the datagraph configuration file. \\
  \end{tabular}
\end{table}

This tag does not have any child tags.

\xmltag{<evaluation>}

This tag contains the equation for QoS evaluation. The text inside 
this tag should be written in a certain way. The variables of this 
equation should be specified with the log format it belongs to. The 
dot operator after the log format specified the actual variable. This tag 
does not have any child tags.

\xmltag{<aggregation>}

This tag specifies a function used to convert a dataset to a single 
value. Some examples of aggregation functions are AVERAGE, 
MIN, MAX and SUM. This tag does not have any child tags.

\xmltag{<grouping>}

For a given aggregation function this tag tell how to classify 
datasets that are independent of each other. This is the 
container tag for such grouping items
The following is a list of child tags: \texttt{<groupitem>}. 

\xmltag{<groupitem>}

This tag represent an actual grouping item. The \textit{name} 
attribute specify the grouping item. This attribute consists of 
a particular log format id and a variable name. Their should be a 
\_ between the log format and the variable. This tag has the 
following attributes. 
\begin{table}
 \begin{tabular}{lcccl}
  \hline
  \textbf{Name} & \textbf{Type} & \textbf{Default Value} & \textbf{Required} & \textbf{Description} \\
  \hline
  name & String  & & Yes & The name of the grouping item. \\
  \end{tabular}
\end{table}

\xmltag{<services>}

UNITE has a facility to show the results of QoS analysis using different 
presentation methods such as graphs. There are many different presentation 
softwares available such as Microsoft Excel, Gnuplot. This tag contains the 
set of such services the testers want present the analyzed results. This tag has 
the following child tags:\texttt{<service>}

\xmltag{<service>}

This tag represent a presentation service. In general these presentation services 
are developed externally to UNITE and plugged in via a common interface.
Therefore child tags of this tag gives the details required to load these services 
during the runtime. This tag has the following attributes.

\begin{table}
 \begin{tabular}{lcccl}
  \hline
  \textbf{Name} & \textbf{Type} & \textbf{Default Value} & \textbf{Required} & \textbf{Description} \\
  \hline
  name & String  & & Yes & The name of the grouping item. \\
  \end{tabular}
\end{table}

\noindent The following is a list of child tags:
\texttt{<location>}, \texttt{<classname>}, \texttt{<params>}.

\xmltag{<location>}

This tag specify the location of a presentation service module. UNITE 
will search for this location when it wants to load the presentation 
service. This tag does not have any child tags.

\xmltag{<classname>}

This tag specify the name of the class of the presentation 
service. This tag does not have any child tags.

\xmltag{<params>}

This tag specify the additional parameters required to invoke 
the presentation service. This tag does not have any child tags.

\section{Invoking UNITE}
\label{sec:unite-invoke}

Assuming the CUTS runtime architecture has been built and installed 
correctly, the CUTS UNITE tool is installed at the following 
location:
\begin{lstlisting}
%> $CUTS\_ROOT/bin/cuts-unite
\end{lstlisting}
To see a complete list of command-line options, use the following 
command:
\begin{lstlisting}
%> $CUTS\_ROOT/bin/cuts-unite --help
\end{lstlisting}

A typical invocation of UNITE tool requires the data file which 
contains the system execution trace and the UNITE configuration 
file discussed above as arguments. The datagraph file location 
is specified inside the UNITE configuration file so it doesn't 
need to be specified as an argument. 
Assuming the configuration discussed above is defined in a file 
named \textit{example.unite},  and the system execution trace is resides 
in a file called \textit{example.cdb} you can run the CUTS UNITE tool with
the configuration via the following command: 
\begin{lstlisting}
%> $CUTS_ROOT/bin/cuts-unite -f example.cdb -c example.unite
\end{lstlisting}