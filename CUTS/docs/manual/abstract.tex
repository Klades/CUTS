% $Id$

\begin{abstract}
Enterprise distributed systems (such as air traffic management systems, 
cloud computing centers, and shipboard computing environments) are steadily 
increasing in size (e.g., lines of source code and number of hosts in the 
target environment) and complexity (e.g., application scenarios). To address 
challenges associated with developing next-generation enterprise distributed 
systems, the level of abstraction for software development steadily increases
as well. Distributed system developers therefore focus more on the system's
``business-logic'' instead of wrestling with low-level implementation details, 
such as development and configuration, resource management, fault tolerance. 
Moreover, increase the level of abstraction for software development helps
promotes reuse of the system's ``business-logic'' across different application 
domains, which inherently reduces (re)invention of core intellectual property.

Although increasing the level of abstaction for software development is improving 
the software lifecycle of next-generation enterprise distributed systems, system 
quality-of-service (QoS) properties (\textit{e.g.}, latency, throughput, and 
scalability) are not evaluated until late in the software lifecycle, \textit{i.e.}, 
during system integration time. This is due in part to the \textit{serialized-phasing 
developmen problem} where the infrastructure- and application-level system entities, 
are developed during different phases of the software lifecycle. Consequently, 
distributed system developers do not realize the system under development does 
not meet its QoS requirements until its too late, \textit{i.e.}, during complete 
system integration time.

System execution modeling (SEM) tools is a model-driven engineering technique
that helps overcome the effects of serialized-phasing development. SEM tools
provide distributed system developers with the necessary artifacts and tools 
for modeling system behavior and workload, such as computational attributes, 
resource requirements, and network communication. The constructed models are
then used to evaluate QoS properties of the system under development during
early phases of the software lifecycle. This enables distributed system 
developers to pinpoint potential performance bottlenecks before they become
to costly to locate and rectify later in the software lifecycle.

The Component Workload Emulator (CoWorkEr) Utilization Test Suite (also known  
as CUTS) is a SEM tool designed for next-generation enterprise distributed 
systems. Distributed system developers and testers use CUTS to model the expected behavior 
and workload of the system components under development using high-level
domain-specific modeling languages. Model interpreters then transform the 
constructed behavior and workload models into source code for their target 
architecture, \textit{i.e.}, the components same interfaces and attributes as 
their real counterpart. Finally, system developers and testers emulate the 
auto-generated components in their target (or representative) environment
and collect QoS metrics. This enables distributed system developers and testers
to conduct system integration test at early stages of software lifecycle,
instead of waiting until complete system integration time to perform such
testing.

\iffalse
Developers and testers can grapically view the collected metrics while the system 
is executing to understand its current performance, and gain insight on how to 
improve it. Developer also have the option of viewing and analyzing collected 
performance metrics postmortem. Lastly, as the real component's implementation 
is completed, it can replace its respective emulation component to produce more 
realistic results and facilitate continuous system integration.
\fi

This book therefore is the end-user's guide to CUTS. It contains information on 
building and installing CUTS, using its domain-specific modeling languages to model
behavior and workload, creating system experiments, and details on CUTS many tools 
that help simpify evaluting QoS of next-generation distributed systems continuously
throughout the software lifecycle.
\end{abstract}
