% $Id$

\chapter{Autobuilds of CUTS}

This section contains detailed instructions for hosting an 
autobuild of CUTS. The autobuild is designed to both build 
and test core functionality of CUTS so errors can be located 
in a timely manner. Moreover, hosting an autobuild is designed
to be an easy process. Please use these instructions as a 
starting point. If you need to change any part of the process,
please feel free to do so. We ask, however, that you please
do not commit your changes back to the repository. 

\section{Hosting a Windows Build}

Windows-based builds of CUTS are conducted using CruiseControl 
(\url{cruisecontrol.sourceforge.net}), which is a Java-based
continuous integration environment. The scripts for hosting a
Windows build are already defined for CUTS. You can find them
at the following location in the CUTS directory structure:
\begin{lstlisting}
%> $CUTS_ROOT/integration
\end{lstlisting}

Once CruiseControl is installed in your environment, you 
can include CUTS in the build cycle in two main steps. First,
you must include the CUTS project in CruiseControl's main 
configuration file (\textit{i.e.}, \texttt{config.xml}). The 
following line is an example of including the Visual Studio
2005 build for CUTS as part of the build cycle.
\begin{lstlisting}
<include.projects 
  file="E:/proj/vc8/CUTS/integration/cruisecontrol/cuts-vc8.config.xml" />
\end{lstlisting}

\textbf{Configuring the build environment.}
After including the desired projects in CruiseControl's main 
configuration file, the next (and final) step is to configure
the environment. This is done by first defining the directory 
structure for the build environment. Each build type (\textit{e.g.},
Visual Studio 2005 and Visual Studio 2008) must have its own
sandbox in the target build environment. The sandbox is where
all projects for that build type must should reside when building
CUTS via CruiseControl. The following listing shows the directory
structure for hosting both a Visual Studio 2005 and Visual Studio 
2008 build on the same machine where the sandboxes are located in
\texttt{E:/proj}.
\begin{lstlisting}
/vc80
  /CUTS
  /ACE_TAO_CIAO
  /XSC
  ...
/vc90
  /CUTS
  /ACE_TAO_CIAO
  /XSC
  ...
\end{lstlisting}

Once you have established the sandbox locations, you must 
inform the build environment of its location. This can be
done by setting one or more of the following environment
variables.
\begin{table}[hbtp]
\centering
\caption{Environment variables that capture the location of a sandbox.}
\begin{tabular}{ll}
    \hline  
    \textbf{Variable Name} & \textbf{Description} \\ 
    \hline
    \texttt{VC80SANDBOX} & Defines the sandbox for Visual Studio 2005 builds \\
    \hline
    \texttt{VC90SANDBOX} & Defines the sandbox for Visual Studio 2008 builds \\
    \hline
  \end{tabular}
\end{table}
Using the example directory structure above, the Visual 
Studio 2005 would be defined as follows:
\begin{lstlisting}
set VC80SANDBOX=E:/proj/vc8
\end{lstlisting}
Notice, this environment variable must be defined in a persistent 
location, such as the user/system environment variables.

Lastly, you must define the appropriate configuration environment
variable. This is necessary because each build environment
may install the required third-party libraries (see 
Appendex~\ref{chap:thirdparty}) in different locations. 
The configuration environment variable(s) therefore points 
to a batch (\texttt{.bat}) file that defines all the environment 
variables need to build CUTS, such as all the environment 
variables defined in Appendex~\ref{chap:thirdparty}. The following 
table is the list of configuration environment variables based
on the build type:
\begin{table}[hbtp]
\centering
\caption{Environment variables that define the build configuration.}
  \begin{tabular}{ll}
    \hline  
    \textbf{Variable Name} & \textbf{Description} \\ 
    \hline
    \texttt{VC80BUILDCFG} & Defines build configuration for Visual Studio 2005 builds \\
    \hline
    \texttt{VC90BUILDCFG} & Defines build configuration for Visual Studio 2008 builds \\
    \hline
  \end{tabular}
\end{table}

For example, if Visual Studio 2005 environment variables are 
defined in \texttt{E:/proj/vc9/setenv.bat}, then \texttt{VC80BUILDCFG} 
would be defined as follows:
\begin{lstlisting}
set VC80BUILDCFG=E:/proj/vc8/setenv.bat
\end{lstlisting}
Notice, this environment variable must be defined in a persistent 
location, such as the user/system environment variables.
\vspace{.2in}

\noindent Congratulations, you are now ready to build 
Windows-based version of CUTS using CruiseControl!