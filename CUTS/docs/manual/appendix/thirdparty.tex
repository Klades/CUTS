% $Id$

\chapter{Building and Installing Third-Party Libraries}
\label{chap:thirdparty}

The following appendix contains best practices for building
and installing third-party libraries used by CUTS projects. It is not
a requirement to install third-party libraries per the instructions in
this appendix. If you are experiencing compilation problems with CUTS
due to third-party libraries, however, the instructions in this appendix
may be of assistance. To make help explain the installation process
of the thirdparty libraries, it will be assumed that all third-party
libraries will be installed at the following location:
\texttt{/opt/thirdparty}.\footnote{This location is representative 
of a common location for installing third-party libraries, and may
be different on your system.}

\section{Boost}
\label{sec:thirdparty-boost}

The Boost libraries are required when building and installing CUTS. All
artifacts for Boost, such as prebuilt versions and source files, can be 
downloaded from the following location: \url{http://www.boost.org}. 

\subsection{Installation}

\subsubsection{Non-Windows}
 
If you are installing Boost on a non-Windows system, \textit{e.g.}, Linux, 
Soloris, or MacOS X, then you can download the source code directly from its
main website (\url{http://www.boost.org}), build it, and install it using
standard procedures.

\subsubsection{Windows}

Installing Boost from sources on Windows is not a trivial process. 
Moreover, the prebuilt version of Boost available for download from 
the main website does not contain the correct configuration or directory
layout. It is therefore recommended that you download and use the prebuilt 
version of Boost available for download from the CUTS website at the 
following location:
\url{http://www.dre.vanderbilt.edu/CUTS/downloads/thirdparty/boost}. Please download
the correct version of Boost that matches your version of Microsoft Visual
C++ and extract it to a common location on disk.

\subsection{System Configuration}

To build CUTS correctly, the Boost environment variables must be defined. Please 
define the following required environment variables:
\begin{lstlisting}
(*@\texttt{\%> export BOOST\_ROOT=} location of Boost@*)
(*@\texttt{\%> export BOOST\_VERSION=} name of subdirectery under \texttt{\$BOOST\_ROOT/include}, \textit{e.g.}, \texttt{boost-1\_34\_1}@*)
(*@\texttt{\%> export LD\_LIBRARY\_PATH=\$LD\_LIBRARY\_PATH:\$BOOST\_ROOT/lib}\footnote{On
Windows-based systems, please add library paths to \texttt{PATH}. On Mac OS-based
systems, please add library paths to \texttt{DYLD\_LIBRARY\_PATH}.}@*)
(*@\texttt{\%> export PATH=\$PATH:\$BOOST\_ROOT/lib}@*)
\end{lstlisting}
In some cases, Boost libraries will have an optional configuration appended to 
their filename, such as \texttt{-mt-d}. To use a specific configuration of 
Boost, the \texttt{BOOST\_CFG} environment variable must be defined. The 
following is an example of defining the Boost configuration environment 
variable:
\begin{lstlisting}
(*@\texttt{\%> export BOOST\_CFG=-mt-d}@*)
\end{lstlisting}
The configuration of Boost is now complete. Once all the other
third-party libraries are installed you will be able to build CUTS
(see Chapter~\ref{chap:install}).

\section{Xerces-C}
\label{sec:thirdparty-xercesc}

Xerces-C 3.x is required when building and installing CUTS and other third-party
libraries. All artifacts for Xerces-C, such as prebuilt versions and source files, 
can be downloaded from the following location: \url{http://xerces.apache.org/xerces-c}. 

\subsection{Installation}

\subsubsection{Non-Windows}
 
If you are installing Xerces-C on a non-Windows system, \textit{e.g.}, Linux, 
Soloris, or MacOS X, then you can download the source code directly from its
main website (\url{http://xerces.apache.org/xerces-c}), build it, and install it using 
standard procedures.

\subsubsection{Windows}

Installing Xerces-C from sources on Windows is not a trivial process. 
Moreover, the prebuilt version of Xerces-C available for download from 
the main website does not contain the correct configuration or directory
layout. It is therefore recommended that you download and use the prebuilt 
version of Xerces-C available for download from the CUTS website at the 
following location:
\url{http://www.dre.vanderbilt.edu/CUTS/downloads/thirdparty/xerces-c}. Please download
the correct version of Xerces-C that matches your version of Microsoft Visual
C++ and extract it to a common location on disk.

\subsection{System Configuration}

To build CUTS correctly, the Xerces-C environment variables must be defined. Please 
define the following required environment variables:
\begin{lstlisting}
(*@\texttt{\%> export XERCESCROOT=} location of Xerces-C@*)
(*@\texttt{\%> export LD\_LIBRARY\_PATH=\$LD\_LIBRARY\_PATH:\$XERCESCROOT/lib}@*)
\end{lstlisting}
The configuration of Xerces-C is now complete. Once all the other
third-party libraries are installed you will be able to build CUTS
(see Chapter~\ref{chap:install}).

\section{DOC Group Middleware}
\label{sec:thirdparty-acetaociao}

DOC Group Middleware is required when building and installing CUTS. All 
artifacts the DOC Group Middleware can be downloaded from the following 
location: \url{http://www.dre.vanderbilt.edu/CIAO}. The following section 
therefore covers building and installing ACE, TAO, CIAO, and DAnCE from
the DOC Group Middleware.

\subsection{System Configuration}

Unlike Boost, you must configure you system installing ACE and TAO. Once 
you have downloaded the ACE and TAO tarball from the website above
\footnote{It is recommended you download either the latest snapshot or beta 
version of ACE and TAO tarball. This will ensure you have the must up-to-date
version of ACE and TAO that is compatible with CUTS.}, 
please define the following environment variables:
\begin{lstlisting}
(*@\texttt{\%> export ACE\_ROOT=} location of ACE@*)
(*@\texttt{\%> export TAO\_ROOT=} location of TAO@*)
(*@\texttt{\%> export CIAO\_ROOT=} location of CIAO@*)
(*@\texttt{\%> export DANCE\_ROOT=\$CIAO\_ROOT/DAnCE}@*)

(*@\texttt{\%> export LD\_LIBRARY\_PATH=\$LD\_LIBRARY\_PATH:\$ACE\_ROOT/lib}@*)
(*@\texttt{\%> export PATH=\$PATH:\$ACE\_ROOT/bin:\$CIAO\_ROOT/bin:\$DANCE\_ROOT/bin}@*)
\end{lstlisting}
After defining the ACE, TAO, and CIAO environment variables, the next step is 
configuring the middleware for the target platform. This is accomplished by 
defining the \texttt{\$ACE\_ROOT/ace/config.h} header file, which will include the correct 
configuration for your platform. The included configuration file has the 
form \texttt{config-*}, where \texttt{*} is the platform of choice.
The following is an example \texttt{config.h} file for building ACE, TAO, and
CIAO on Linux platforms:
\begin{lstlisting}[language=C++]
// -*- C++ -*-

#ifndef _ACE_CONFIG_H_
#define _ACE_CONFIG_H_

#include ``config-linux.h''

#endif // !defined _ACE_CONFIG_H_
\end{lstlisting}

\subsubsection{Non-Windows}
Configuring ACE, TAO, and CIAO on non-Windows systems requires one more 
step. The configuration for the build engine also needs to be defined. 
Please use the following step to define the build engine's configuration:
\begin{lstlisting}
(*@\texttt{\%> cd \$ACE\_ROOT/include/makeinclude}@*)
(*@\texttt{\%> ln -s platform\_*.GNU platform\_macros.GNU}@*)
\end{lstlisting}
where \texttt{*} is the architecture for your platform of choice.

\subsection{Installation}

After you have configured your system for ACE, TAO, CIAO, and DAnCE, you 
are ready to build and install ACE and TAO. Please use the following steps 
to build and install ACE, TAO, CIAO, and DAnCE:
\begin{lstlisting}
(*@\texttt{\%> cd \$CIAO\_ROOT}@*)
(*@\texttt{\%> \$ACE\_ROOT/bin/mwc.pl -type [build tool] CIAO\_TAO\_DAnCE.mwc}@*)
\end{lstlisting}
This will generate the workspace for building and installing ACE, TAO, CIAO,
and DAnCE. The name of the workspace depends on the build tool specified
when generating the workspace. Finally, using your build tool of choice, 
you can build and install ACE, TAO, CIAO, and DAnCE.

\section{XML Schema Compiler (XSC)}
\label{sec:thirdparty-xsc}

XSC is required when building and installing CUTS. All artifacts
for XSC can be downloaded from the following location: 
\url{svn://svn.dre.vanderbilt.edu/XSC/trunk}. The remainder of this section
discusses how to build and install XSC.

\subsection{System Configuration}

You must configure you system installing XSC. Once you have downloaded 
XSC from the website above please define the following environment variables:
\begin{lstlisting}
(*@\texttt{\%> svn co svn://svn.dre.vanderbilt.edu/XSC/trunk XSC}@*)
(*@\texttt{\%> export XSC\_ROOT=} location of XSC@*)
(*@\texttt{\%> export LD\_LIBRARY\_PATH=\$LD\_LIBRARY\_PATH:\$XSC\_ROOT/lib}@*)
(*@\texttt{\%> export PATH=\$PATH:\$XSC\_ROOT/bin}@*)
\end{lstlisting}

\subsection{Installation}

After you have configured your system for XSC, you are ready to build and 
install it. Please use the following steps to build and install XSC:
\begin{lstlisting}
(*@\texttt{\%> cd \$XSC\_ROOT}@*)
(*@\texttt{\%> \$ACE\_ROOT/bin/mwc.pl -type [build tool] -features xsc=1,xerces3=1 XSC.mwc}@*)
\end{lstlisting}
This will generate the workspace for building and installing XSC. The name of 
the workspace depends on the build tool specified when generating the 
workspace. Finally, using your build tool of choice, you can build and 
install XSC.

\section{Perl-Compatible Regular Expressions (PCRE)}
\label{sec:thirdparty-pcre}

PCRE is required when building and installing several projects in CUTS. 
You can download the source code for PCRE from the following 
location: \url{http://www.pcre.org}

\subsection{Installation}

\subsubsection{Non-Windows}
 
If you are installing PCRE on a non-Windows system, \textit{e.g.}, Linux, 
Soloris, or MacOS X, then you can download the source code directly from its
main website (\url{http://www.pcre.org}), build it, and install it using
standard procedures.

\subsubsection{Windows}

Installing PCRE from sources on Windows is not a trivial process. 
It is therefore recommended that you download and use the prebuilt 
version of PCRE available for download from the CUTS website at the 
following location:
\url{http://www.dre.vanderbilt.edu/CUTS/downloads/thirdparty/pcre}. Please download
the correct version of PCRE that matches your version of Microsoft Visual
C++ and extract it to a common location on disk.

\subsection{System Configuration}

To build CUTS correctly, the PCRE environment variables must be defined. Please 
define the following required environment variables:
\begin{lstlisting}
(*@\texttt{\%> export PCRE\_ROOT=} location of PCRE@*)
(*@\texttt{\%> export PCRE\_VERSION=} version of PCRE, blank if not applicable@*)
(*@\texttt{\%> export LD\_LIBRARY\_PATH=\$LD\_LIBRARY\_PATH:\$PCRE\_ROOT/lib}@*)
\end{lstlisting}
The configuration of PCRE is now complete. Once all the other
third-party libraries are installed you will be able to build CUTS
(see Chapter~\ref{chap:install}).

\section{SQLite}
\label{sec:thirdparty-sqlite}

SQLite is required when building and installing several 
projects in CUTS. You can download the source code for SQLite
from the following location: \url{http://www.sqlite.org}

\subsection{Installation}

\subsubsection{Non-Windows}
 
If you are installing SQLite on a non-Windows system, \textit{e.g.}, Linux, 
Soloris, or MacOS X, then you can download the source code directly from its
main website (\url{http://www.sqlite.org}), build it, and install it using  
standard procedures.

\subsubsection{Windows}

Installing SQLite from sources on Windows is not a trivial process. 
It is therefore recommended that you download and use the prebuilt 
version of SQLite available for download from the CUTS website at the 
following location:
\url{http://www.dre.vanderbilt.edu/CUTS/downloads/thirdparty/sqlite}. Please download
the correct version of SQLite that matches your version of Microsoft Visual
C++ and extract it to a common location on disk.

\subsection{System Configuration}

To build CUTS correctly, the SQLite environment variables must be defined. Please 
define the following required environment variables:
\begin{lstlisting}
(*@\texttt{\%> export SQLITE\_ROOT=} location of SQLite@*)
(*@\texttt{\%> export LD\_LIBRARY\_PATH=\$LD\_LIBRARY\_PATH:\$SQLITE\_ROOT/lib}@*)
(*@\texttt{\%> export PATH=\$PATH:\$SQLITE\_ROOT/bin}@*)
\end{lstlisting}
The configuration of SQLite is now complete. Once all the other
third-party libraries are installed you will be able to build CUTS
(see Chapter~\ref{chap:install}).

