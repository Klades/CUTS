% $Id$

\chapter{CUTS Node Management Facilities}
\label{chap:node}

This chapter discusses the CUTS node manager facilities. These
facilities are designed to simplify management of many processes
that must execute on many different nodes in a large-scale distributed
testing environment. Furthermore, these facilities simplify management
of many different execution environments, which is hard to do using 
traditional shell scripting facilities, such as Bash, C-Shell, and
Batch files on Windows.

\section{Overview}
\label{sec:node-overview}

Distributed testing and execution environments, such as network 
testbeds, consist of many nodes. These nodes are responsible for 
executing many different processes that must support the overall
goals of the execution environment. For example, nodes in the 
execution environment may host processes that manage software components,  
processes that synchronize clocks, and processes that collect
software/hardware instrumentation data.

When dealing with many processes that must execute on a given node,
the traditional approach is to create a script that launches all
processes for that given node. For example, in *Unix environments,
Bash scripts can be used to launch a set of processes. The advantage
of using such scripts is that it provides a \textit{repeatable}
technique for ensuring that all process for a given node are launched,
and done so in the correct order. Moreover, the script is persistent
and can be easily updated to either add or remove processes as needed.

Although such scripts are sufficient when working in a distributed 
testing environment, its approach adds more complexity to the 
distributed testing process. First, the scripts are only repeatable
after \textit{manually} invoking the script. This means that 
distributed testers much manually connect to each host in the 
environment and (re)start the script. Even if it is possible to 
remotely connect to the host via automation to (re)start the 
script, it is still \textit{hard} to know what processes are
associated with a given script.

Secondly, such scripts do no easily support injection/removal
of processes on-the-fly. It is assumed in the script used to 
configure the host. At that point, all processes injected/removed
from the host are assumed to be part of all the processes on 
that machine. Lastly, it is \textit{hard} to switch between
different execution environment where an execution environment
is set of environment variables and a set of processes.

To address the limitations of traditional approaches to executing
and managing a distributed testing environment, CUTS provides
node management facilities. The facilities provide methods for 
managing many different execution environments, which can easily 
be swapped between. The facilities also provide tools for remotely
controlling nodes in the execution environment. 

The remainder of the chapter therefore presents the basics for
using the CUTS node management facilities in a production-like 
environment.In particular, the following information is covered 
in this chapter:
\begin{itemize}
  \item \textbf{CUTS Node Daemon} - The CUTS Node Daemon for 
  managing many different execution environments for a node. 
  Details of the CUTS Node Daemon are presented in Section~\ref{sec:node-daemon}.

  \item \textbf{CUTS Node Daemon Client} - The CUTS Node Daemon Client 
  is responsible remotely controlling the CUTS Node Daemon. Details of 
  the CUTS Node Daemon Client are provided in 
  Section~\ref{sec:node-daemon-client}.
\end{itemize}

\section{The Node Daemon}
\label{sec:node-daemon}

The CUTS Node Daemon is responsible for managing different execution
environments for a node. There is typically one CUTS Node Daemon executing
on a single node. There can be cases where more than one CUTS
Node Daemon executes on the same machine, but this defeats the purpose
of the daemon. Regardless of the situation, what is discussed in the
remainder of this chapter can be used to execute one or more 
CUTS Node Daemon processes of a single host.

\subsection{Configuring the Node Daemon}
\label{sec:node-daemon-config}

The CUTS Node Daemon is designed to manage different execution 
environments. Before it can doe this, however, it must be properly
configured. The configuration for the CUTS Node Daemon is similar
to a shell script, except that it is specified in XML format. 
Listing~\ref{listing:cutsnode.config} illustrates an example
configuration for the CUTS Node Daemon.
\lstinputlisting[label=listing:cutsnode.config,
  caption=An example configuration file for the CUTS Node Daemon.,
  captionpos=b,numbers=left]{runtime/cutsnode.config}

Without going into the full details of the configuration in
Listing~\ref{listing:cutsnode.config}, the CUTS Node Configuration
contains three critical sections:
\begin{itemize}
  \item \textbf{Environment} -- The environment section, denoted 
  by the \texttt{<environment>} tag, defines an execution environment. 
  There can be more than one execution environment in a configuration
  As shown in Listing~\ref{listing:cutsnode.config}, there is only
  one execution environment named \texttt{default}. The remaining 
  details of the execution environment is discussed in 
  Section~\ref{sec:node-daemon-schema}.
  
  \item \textbf{Variables} -- The variables section, denoted
  by the \texttt{<variable>} tag, defines environment variables for the 
  parent execution environment. All processes in the execution environment 
  can use these environment variables via the \$\{\} tag (see 
  line~\ref{line:useenv}). As shown
  in Listing~\ref{listing:cutsnode.config}, the \texttt{default}
  execution environment contains one environment variable named
  \texttt{NameServiceIOR}. This environment variable is not 
  explicitly used in the configuration. Instead, the processes
  in this configuration expect for the variable to be defined
  as part of the default environment variables. The 
  remaining defaults of the variables section is discussed
  in Section~\ref{sec:node-daemon-schema}.

  \item \textbf{Startup/Shutdown Processes} -- The startup/shutdown section, 
  denoted by \texttt{<startup>} and \texttt{<shutdown>}, 
  respectively, defines the processes to execute when an execution
  environment starts up or shuts down. As shown in
  Listing~\ref{listing:cutsnode.config}, the configuration defines
  only a set of startup processes. The remaining details of the 
  processes section is discussed in Section~\ref{sec:node-daemon-schema}.
\end{itemize}

\subsection{Running the Node Daemon}
\label{sec:node-daemon-config}

Assuming the CUTS runtime architecture has been built and installed 
correctly, the CUTS Node Daemon is installed at the following 
location:
\begin{lstlisting}
%> $CUTS_ROOT/bin/cutsnode_d
\end{lstlisting}
To see a complete list of command-line options, use the following 
command:
\begin{lstlisting}
%> $CUTS_ROOT/bin/cutsnode_d --help
\end{lstlisting}
Assuming the configuration discussed above is defined in a file 
named \textit{cutsnode.config}, you run the CUTS Node Daemon with
the configuration via the following command: 
\begin{lstlisting}
%> $CUTS_ROOT/bin/cutsnode_d -c cutsnode.config
\end{lstlisting}

\subsection{Configuring the Node Daemon Server}
\label{sec:node-daemon-server}

When you run the CUTS Node Daemon, the server is configured with 
its default values. In many cases, you may want to change
the server's configuration. For example, you may want to bind the
server to localhost instead of its hostname, or you may want to change
the port on which it is listening.

You can configure the server using the \texttt{--server-args}
command-line option. This should be a quoted command-line option because
there can many different arguments passed to the server. The following
table lists the different command-line options:
\begin{table}[h]
  \begin{tabular}{lcl}
  \hline
  \textbf{Name} & \textbf{Required} & \textbf{Description} \\
  \hline
  -\--register-with-iortable & No & Name used to expose server via the IORTable \\
  -\--register-with-ns & No & Name used to register server with naming service \\
  \end{tabular}
\end{table}

In addition to these options, you can passed any option supported
by the ORB, such as \texttt{-ORBEndpoint}. The following command-line
illustrates how to change to listening address of the CUTS Node Daemon
to \texttt{node.hydra.iupui.edu:20000} and expose it via the IORTable
as \texttt{CUTS/NodeDaemon}.
\begin{lstlisting}
%> $CUTS_ROOT/bin/cutsnode 
  --server-args="-ORBEndpoint iiop://node.hydra.iupui.edu:20000
  --register-with-iortable=CUTS/NodeDaemon"
\end{lstlisting}

\subsection{Node Daemon Schema Definition}
\label{sec:node-daemon-schema}

The section discusses the details of the CUTS Node Daemon 
configuration based on its underlying XML Schema Definition.

\xmltag{<cuts:node>}

This is the main tag for the configuration document. Its XML namespace
must have the \url{http://cuts.cs.iupui.edu} definition. If it does
not have this definition, then the configuration will not load.

\xmltag{<environment>}

This tag defines the different execution environments. As previously
stated, there can be more than one execution environment in a 
configuration file. The following is a list of attributes supported
by this tag.
\begin{table}[h]
  \begin{tabular}{lcccl}
  \hline
  \textbf{Name} & \textbf{Type} & \textbf{Default Value} & \textbf{Required} & \textbf{Description} \\
  \hline
  id & String  & & Yes & Unique id for the execution environment \\
  inherit & Boolean & false & No & Inherit the system's environment variables \\
  active & Boolean & false & No & This is the active execution environment \\
  \end{tabular}
\end{table}

\noindent The following is a list of child tags:
\texttt{<variables>}, \texttt{<startup>}, \texttt{<shutdown>}.

\xmltag{<variables>}

This is the container tag for the environment variables in execution 
environment. This tag does not contain any attributes. The following 
is a list of child tags: \texttt{<variable>}.

\xmltag{<variable>}

This tag is used to defined environment variables for an execution
environment. The following is a list of attributes supported
by this tag.
\begin{table}[h]
  \begin{tabular}{lcccl}
  \hline
  \textbf{Name} & \textbf{Type} & \textbf{Default Value} & \textbf{Required} & \textbf{Description} \\
  \hline
  name & String  & & Yes & Name of the environment variable \\
  value & String & & Yes & Value of the environment variable \\
  \end{tabular}
\end{table}

\noindent This tag does not contain any child tags.

\xmltag{<startup>}

This tag is a container for a list of processes that are to 
executed with an environment starts up. The order in which 
the processes are listed is their startup order. The following 
is a list of child tags: \texttt{<process>}.

\xmltag{<shutdown>}

This tag is a container for a list of processes that are to 
executed with an environment is shutdown. The order in which 
the processes are listed is their shutdown order. The following 
is a list of child tags: \texttt{<process>}.

\xmltag{<process>}

This tag defines an executable process for the execution
environment. The following is a list of attributes supported
by this tag.
\begin{table}[h]
  \begin{tabular}{lcccl}
  \hline
  \textbf{Name} & \textbf{Type} & \textbf{Default Value} & \textbf{Required} & \textbf{Description} \\
  \hline
  id & String  & & Yes & Unique id for the process \\
  delay & Float & 0.0 & No & Seconds to wait before executing process \\
  waitforcompletion & Boolean & false & No & Wait for process to complete before progressing \\
  \end{tabular}
\end{table}

\noindent The following is a list of child tags:
\texttt{<executable>}, \texttt{<arguments>}, \texttt{<workingdirectory>}.

\xmltag{<executable>}

This tag defines the complete/relative path of the executable process.
Environment variables can be used within this tag. This tag does not
contain any child tags.

\xmltag{<arguments>}

This tag defines the command-line for the executable process.
Environment variables can be used within this tag. This tag does not
contain any child tags.

\xmltag{<workingdirectory>}

This tag defines the working directory for the executable process 
(\textit{i.e.}, the directory where executable process will be spawned).
Environment variables can be used within this tag. This tag does not
contain any child tags.

\section{The Node Daemon Client}
\label{sec:node-daemon-client}

The purpose of the CUTS Node Daemon Client is to remotely control
different CUTS Node Daemon application. This prevents users from
having to manually login to different nodes when there is need to 
alter the execution environment.

\subsection{Running the Node Daemon Client}
\label{sec:node-daemon-config}

Assuming the CUTS runtime architecture has been built and installed 
correctly, the CUTS Node Daemon Client is installed at the following 
location:
\begin{lstlisting}
%> \$CUTS\_ROOT/bin/cutsnode
\end{lstlisting}
To see a complete list of command-line options, use the following 
command:
\begin{lstlisting}
%> \$CUTS\_ROOT/bin/cutsnode --help
\end{lstlisting}
Assuming the CUTS Node Daemon is running at the address 
\texttt{iiop://node.hydra.iupui.edu:20000} and registered with 
the IORTable as \texttt{CUTS/NodeDaemon} as discussed above,
then you can connect and control it via the CUTS Node
Daemon Client using the following command: 
\begin{lstlisting}
%> $CUTS_ROOT/bin/cutsnode --reset -ORBInitRef 
  NodeDaemon=corbaloc:iiop:node.hydra.iupui.edu:20000/CUTS/NodeDaemon
\end{lstlisting}
In this example, the \texttt{-ORBInitRef} is a \texttt{corbaloc:}. This,
however, need not always be the case. It can be any StringIOR support by
\textit{The ACE ORB (TAO)}, such as a IOR file or naming service location.

