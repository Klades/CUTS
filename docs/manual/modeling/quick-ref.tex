% $Id$

\newenvironment{modelelement}[3]{
\begin{figure}[h!]
  \includegraphics[scale=#2]{#3}
\end{figure}
\subsection*{#1}
}{\begin{center}
\line(1,0){300}
\end{center}}

\chapter{PICML Modeling Quick Reference}
\label{sec:modeling-quick-ref}

This section provides a quick reference to the modeling
environment. In particular, it focuses on key elements in 
PICML modeling language (\url{http://www.dre.vanderbilt.edu/cosmic}) 
so you do not spend much time and
effort trying to determine what elements you need to build
a CUTS model.

The layout of this section is as follows: each subsection
corresponds to a folder type under the root folder in a 
PICML model. Within each section, we then highlight the 
important elements, \textit{i.e.}, those that are used most
often. This does not necessarily mean you will not use elements
covered in the section. Finally, for each element, we explain
its purpose so you have a better understanding of its role 
in the model.

\section{Predefined Types}

Predefined types, \textit{e.g.}, integer, boolean, and string, 
are defined in the \texttt{PredefinedTypes} folder. When you 
add a new predefined type folder to the model, PICML will 
automatically populate it will all the predefined types it 
currently supports.

\section{Interface Definitions}

The \texttt{InterfaceDefinitions} folder is used to define the 
interfaces, complex data types, and components, of the distributed
system. This folder can contain only the \texttt{File} element
type, which represents a \textit{Interface Definition File (IDL)}
file. Within the \texttt{File} element, the following elements are 
the most import elements when constructing a model that is useable
by CUTS (listed in alphabetical order):

\begin{modelelement}{Aggregate}{0.2}{modeling/figs/aggregate.png}
The \texttt{Aggregate} is similar to the structure construct (\textit{i.e.}, 
\texttt{struct}) in C and C++, or a tuple in dynamic languages. It 
contains members elements, which reference their data type (\textit{e.g.}, 
\texttt{String}, \texttt{LongInteger}, and \texttt{DateTime}). By
default, all members of the aggregate are considered public.

The following is a list of the aggregate's child elements that
are most important, which are discussed in this section: 
\texttt{Member} and \texttt{Key}.
\end{modelelement}

\begin{modelelement}{Component}{0.6}{modeling/figs/component.png}
The \texttt{Component} element is an abstraction that provides/\-requires 
a set of services or sends/\-receives events. This is determined by
the ports it exposes to the other components. The component can
also contain attributes, which are points-of-configuration that
whose value is determined at deployment time. In CUTS, the component 
is where you model application behavior for early integration
testing via emulation. 

The following is a list of the component's child elements that
are most important, which are discussed in this section: 
\texttt{ProvidedRequestPort}, \texttt{RequiredRequestPort},
\texttt{InEventPort}, \texttt{OutEventPort}, and \texttt{At\-tri\-bute},
and \texttt{Read\-onlyAt\-tri\-bute}.
\end{modelelement}

\clearpage

\begin{modelelement}{Event}{0.2}{modeling/figs/event.png}
The \texttt{Event} element is similar to an aggregate in PICML in 
that it is a collection of data types. It can also contain 
more complex definitions, which are currently ignored in
CUTS. The main difference is that an event is used to send
data between two components via their \texttt{InEventPort}
and \texttt{Out\-Event\-Port} ports. 

The following is a 
list of an event's child elements that are most important, 
which are discussed in this section: \texttt{Member}.
\end{modelelement}

\begin{modelelement}{InEventPort}{0.6}{modeling/figs/ineventport.png}
The \texttt{InEventPort} element is a child of the \texttt{Component}
element that represents a port for receiving events from other components.
It is a reference element that refers to any \texttt{Event} element in
the model.
\end{modelelement}

\begin{modelelement}{Key}{1.0}{modeling/figs/key.png}
The \texttt{Key} element is a child element of an aggregate element. 
It is
primarily used in DDS to define the key value(s) of a data
type. If you are using the DDS code generators in CUTS, then
you will definitely need to make use of this element. It
is the only way to ensure you DDS data type definitions
are complete.
\end{modelelement}

\begin{modelelement}{Member}{0.2}{modeling/figs/member.png}
The \texttt{Member} element is a child element of an aggregate 
and event. It
is a reference element that points to a members data type.
\end{modelelement}

\begin{modelelement}{Object}{0.7}{modeling/figs/object.png}
The \texttt{Object} element is representative of an interface
in a distributed system. It can also be thought of as
a remote interface. This means that the object can have
methods, which are invokable by other components. Currently,
CUTS does not associate any behavior with objects. 
\end{modelelement}

\begin{modelelement}{OutEventPort}{0.6}{modeling/figs/outeventport.png}
The \texttt{OutEventPort} element is a child of the \texttt{Component}
element that represents a port for sending events to other components.
It is a reference element that refers to any \texttt{Event} element in
the model.
\end{modelelement}

\clearpage

\begin{modelelement}{Package}{0.2}{modeling/figs/package.png}
The \texttt{Package} element is representative of a 
namespace is C++ and packages in Java. It is an object
that provide scope hierarchy to the definition. The 
package contains the same elements as a file. This 
means the important elements discussed in this section
also apply to a package element. 
\end{modelelement}

\begin{modelelement}{ProvidedRequestPort}{0.6}{modeling/figs/providedrequestport.png}
The \texttt{ProvidedRequestPort} element is a child of the \texttt{Component}
element that represents a service that is provide by the parent component.
It is a reference element that refers to any \texttt{Object} element in
the model.
\end{modelelement}

\begin{modelelement}{RequiredRequestPort}{0.6}{modeling/figs/requiredrequestport.png}
The \texttt{RequiredRequestPort} element is a child of the \texttt{Component}
element that represents a service that is required by the parent component.
It is a reference element that refers to any \texttt{Object} element in
the model.
\end{modelelement}

\section{Component Implementations}

The \texttt{ComponentImplementations} folder is used to define 
different component assemblies (\textit{i.e.}, set of inter-connected
components that communicate with each other).
This folder can contain only the \texttt{ComponentImplementationContainer} 
element type, which is just a container with a view for visualizing
a contain component assembly. Because of how component assemblies are
defined in PICML, the \texttt{ComponentImplemenationContainer} has
many different elements. Of those elements, the following are most 
important when constructing a model that is useable by CUTS 
(listed in alphabetical order):

\begin{modelelement}{ComponentAssembly}{0.6}{modeling/figs/componentassembly.png}
The \texttt{ComponentAssembly} model element is a container class that
contains a set of related \texttt{ComponentInstance} elements and their
connections. Within a component assembly, it is possible to instantiate
a component assembly. This is done by, first, create a standard component
assembly. In another component assembly, insert the former component 
assembly as a GME model instance.
\end{modelelement}

\begin{modelelement}{ComponentInstance}{0.6}{modeling/figs/component.png}
The \texttt{ComponentInstance} model element is used to instantiate an
instance of a \texttt{Component} implementation, which is defined in another
part of the model. When you insert a \texttt{ComponentInstance} into the
model, PICML's model intelligence helps you select the component instance's
correct implementation.
\end{modelelement}

\begin{modelelement}{Property}{0.2}{modeling/figs/property.png}
The \texttt{Property} element (in the \texttt{DeploymentProperties}
aspect) is used to define configuration
settings for elements in the model. In the component assembly, the
property element can connect to an \texttt{Attribute}, 
\texttt{ComponentAssembly}, \texttt{ComponentInstance}, and
\texttt{ReadonlyAttribute}.

In the component assembly, the following property names have
predefined meanings:
\begin{table}[h]
  \centering
  \begin{tabular}{lcl}
  \hline
  \textbf{Property Name} & \textbf{Data Type} & \textbf{Description}\\
  \hline
  InstanceIOR & String & Set the output location of IOR \\
  RegisterNaming & String & Register component with naming service  \\
  LocalityArguments & String & Set the parameters for the locality manager \\
  CPUAffinity & String & Set the CPU affinity for the deployment \\
  ProcessPriority & String & Set the processor priority for the deployment \\
  \end{tabular}

  \caption{Listing of reserved names for Property elements within
  the component assembly aspect.}
  \label{table:deployment-properties}
\end{table}

\end{modelelement}

\section{Targets}

The \texttt{Targets} folder is used to define different network
configurations and topologies of the target execution environment. 
This folder can contain only the \texttt{Domain} element
type, which a single execution domain (\textit{i.e.}, set of 
connected network entities). Within the \texttt{Target} element, the 
following elements are the most import elements when constructing 
a model that is useable by CUTS (listed in alphabetical order):

\begin{modelelement}{Node}{0.2}{modeling/figs/node.png}
The \texttt{Node} element represents a host in the target network. 
If you want to use a node in a deployment plan, then it must be 
defined here first.
\end{modelelement}

\section{Deployment Plans}

The \texttt{DeploymentPlans} folder is used to different
deployment plans of the model system. This folder can contain 
only the \texttt{DeploymentPlan} element
type, which is a single deployment of either the entire system or
a subset of the system's components. Within the \texttt{DeploymentPlan} 
element, the  following elements are the most import elements when 
constructing a model that is useable by CUTS (listed in alphabetical order):

\begin{modelelement}{CollocationGroup}{0.2}{modeling/figs/collocationgroup.png}
The \texttt{CollocationGroup} element represents a container/\-process 
that hosts a set of components that share the same execution environment,
such as threading model, run-time priorities, and programming language.
This element is a GME set. This means you need to switch to set mode
in order to add deployed model elements (\textit{i.e.}, components
and component assemblies) to a collocation group.
\end{modelelement}

\begin{modelelement}{ComponentAssemblyRef}{0.6}{modeling/figs/componentassemblyref.png}
The \texttt{ComponentAssemblyRef} element is used to add a component
assembly to a deployment plan. This means that all components in the 
assembly are deployed to the node that the component assembly is deployed. 
The element is a GME reference and must refer to a \texttt{ComponentAssembly} 
element.  Finally, you specify what process the component assembly
is deployed to by adding it to a collocation group.
\end{modelelement}

\begin{modelelement}{ComponentInstanceRef}{0.6}{modeling/figs/componentref.png}
The \texttt{ComponentInstanceRef} element is used to add a component
instance to a deployment plan. The element is a GME reference and 
must refer to a \texttt{ComponentInstance} element in a 
\texttt{ComponentAssembly} model. Finally, you specify what process
the component instance is deployed to by adding it to a collocation 
group.
\end{modelelement}

\begin{modelelement}{NodeRef}{0.2}{modeling/figs/node.png}
The \texttt{NodeRef} element represents a host that is part of the
parent deployment. This element is a GME reference that refers to
a \texttt{Node} in the \texttt{Domain} model. Finally, the 
node reference connects with a collocation group to show what
processes/containers are executing on a node in the target
environment.
\end{modelelement}

\begin{modelelement}{Property}{0.2}{modeling/figs/property.png}
The \texttt{Property} element (in the \texttt{DeploymentProperties}
aspect) is is used to define configuration
settings for elements in the model. In the deployment plan, the
property element can connect to a \texttt{NodeRef} and a 
\texttt{CollocationGroup}.

In the deployment plan, the following property names have
predefined meanings:
\begin{table}[htbp]
  \centering
  \begin{tabular}{lcl}
  \hline
  \textbf{Property Name} & \textbf{Data Type} & \textbf{Description}\\
  \hline
  StringIOR & String & Resolveable location of the node manager \\
  \end{tabular}

  \caption{Listing of reserved names for Property elements with the deployment
  plan aspect of the model.}
  \label{table:deployment-properties}
\end{table}
\end{modelelement}
